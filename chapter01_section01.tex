% \chapter{はじめに}
\nocite{BC04873022}
\nocite{Noro2005ca}
\section{研究の背景}
計算機代数分野における重要な計算の一つに連立代数方程式の求解がある.一変数多項式系においては,Gauss消去法のように行列の掃き出しの計算へと帰着することができる.しかし,多変数多項式系においては多項式の割り算が一意に行えないため,項の順序を導入し多変数に拡張した割り算を定義する必要がある.これらによって定められる,多変数の多項式環における割り算を一意に行うことのできるイデアルの生成系を\groebner{}基底という.\groebner{}基底は他にも多くの重要な性質を持っていることから,連立代数方程式の求解だけでなく,代数統計学・代数幾何学・組合せ論などの計算機代数の分野を越えて幅広く応用がなされている.
\par
多項式集合$F$が与えられイデアル$\ideal{F}$の\groebner{}基底の計算を行う際,Buchbergerアルゴリズム\cite{buchberger2006bruno}などにより項順序を固定した上で,\groebner{}基底を得るための計算を行うのが一般的である.
しかし,以下の例のように,計算を行う前に$F$がそのまま\groebner{}基底であるような項順序を得ることができる場合もある.
\begin{example}
	以下の多項式集合$F$は,$z \succ y \succ x$の全次数辞書式順序及び全次数逆辞書式順序においてイデアル$I = \ideal{F}$の\groebner{}基底となっている.
	$$F = \left\{ 2xy + yz, \; x^2 + y + z \right\} \subset \mathbb{C}[x, y, z]$$
\end{example}
このような項順序を見つけることができれば,従来の計算を行わずに\groebner{}基底を得ることができる.
また,ここで得た項順序を,FGLMアルゴリズム\cite{faugere1993efficient}や\groebner{} walk\cite{collart1993grobner}などに代表されるchange of orderingのアルゴリズムによって変換することで,任意の項順序での\groebner{}基底を得ることも可能であり,有効な計算手段となることが考えられる.
\par
例えば,多変数の連立代数方程式の解を求めるためには,多くの場合で辞書式順序(一般的には消去順序)での\groebner{}基底が必要となるが,辞書式順序での計算は遅くなることが知られており,入力の多項式集合の大きさによっては莫大な時間がかかってしまう可能性も否定できない.そこで,多項式集合がそのまま\groebner{}基底となっているような項順序が存在していれば,その項順序を求めたあとにchange of orderingにより辞書式順序の\groebner{}基底を求めることができる.これらの$2$つの計算の計算量が,元々行おうとしていた\groebner{}基底計算の計算量に比べて少なくなっているのであれば,この計算は有用な計算であったと言える.
\par
このように「そのまま\groebner{}基底である」ような項順序を検出する問題は,\emph{\groebner{} basis detection}\cite{gritzmann1993minkowski}や\emph{structural \groebner{} basis detection}\cite{sturmfels1997structural}という名前が付けられており,何れもSturmfelsらによって解かれた既知の問題である.
\par
一方で,パラメータを伴った多項式環において,場合分けされたパラメータ空間と,それぞれに対応する\groebner{}基底を組にしたものを\textbf{包括的\groebner{}基底系}\cite{weispfenning1992comprehensive}と呼ぶ.包括的\groebner{}基底系は,\groebner{}基底計算の中でS多項式の計算が行われる際に,パラメータの付いた係数が$0$か否かで場合分けを行い項を確定させることで得ることができるが,現在はより効率的な方法も考案されている\cite{nabeshima2007speed, kapur2010new}.
\par
これら$2$つの議論を踏まえ,本論文では,(structural) \groebner{} basis detectionにおいて,問題の設定をパラメータを伴った多項式環へと拡張し,それに付随して発見された定理についても取り上げる.
まず第\ref{chapter02:chapter_num}章では,Sturmfelsらとは違ったアプローチでstructural \groebner{} basis detectionの問題を捉え,そこで新たに発見された定理とアルゴリズムを紹介する.
次に,第\ref{chapter03:chapter_num}章では,\groebner{} basis detectionとstructural \groebner{} basis detectionについて,既に知られている部分について述べる.
第\ref{chapter04:chapter_num}章では,これらの問題をパラメータを伴った多項式環へと拡張するために,パラメータ空間を分割するためのアルゴリズムの直接的な方法と,分割の効率化のための方法について記す.