% \chapter{はじめに}
\section{背景}
多項式集合$F$が与えられ\groebner{}基底の計算を行う際,Buchbergerアルゴリズムなどによりイデアルと項順序を固定し,\groebner{}基底を得るための計算を行うのが一般的である.
しかし,計算を行う前に$F$がそのまま\groebner{}基底であるような項順序を得ることができれば,従来の計算を行わずに\groebner{}基底を得ることができる.また,ここで得た項順序をchange of orderingのアルゴリズムによって変換することで,任意の項順序での\groebner{}基底を得ることも可能である.
\begin{theorem}[\groebner{}基底{\cite[theorem1]{buchberger2006bruno}}]
	\groebner{}基底$F = f_1, f_2$
\end{theorem}