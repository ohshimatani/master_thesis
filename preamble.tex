% preamble for "jsartcle"

\documentclass[a4paper, dvipdfmx, 9truept, 9trueptj]{jsreport}
\usepackage[top=30truemm,bottom=30truemm,left=25truemm,right=25truemm]{geometry}

\usepackage{amsmath, amssymb, amsthm}
% \usepackage{ascmac}
\usepackage{mathtools}
\mathtoolsset{showonlyrefs=true}
\usepackage{bm}
\usepackage{graphicx}

\usepackage{cancel}
\usepackage{mathrsfs}
\usepackage{algorithm}
\usepackage{algpseudocode}



%-------定理環境のカスタマイズ

% \newtheoremstyle{mystyle}%   % スタイル名
%     {3truept}%                   % 上部スペース
%     {3truept}%                   % 下部スペース
%     {\itshape}%              % 本文フォント
%     {9truept}%               % 1行目のインデント量
%     {\bfseries}%             % 見出しフォント
%     {.}%                     % 見出し後の句読点
%     {\\}%                      % 見出し後のスペース
%     {}%                      % 見出しの書式(後述)
\newtheoremstyle{break}
  {\topsep}{\topsep}%
  {}{}%
  {\bfseries}{.}%
  {\newline} {\thmname{#1}\thmnumber{ #2}\thmnote{ \bfseries{(#3)}}}%
% \@addtoreset{theorem}{subsection}
%以下、定理環境の日本語化
\theoremstyle{break}
\newtheorem{theorem}{定理}
% \newtheorem*{theorem*}{定理}
\newtheorem{definition}[theorem]{定義}
% \newtheorem*{definition*}[theorem]{定義}
\newtheorem{lemma}[theorem]{補題}
\newtheorem{corollary}[theorem]{系}
\newtheorem{proposition}[theorem]{命題}% jsarticleのみ
\newtheorem{example}[theorem]{例}
\newtheorem*{example*}{例}
\newtheorem{problem}[theorem]{問題}
\newtheorem{solution}[theorem]{解}
\newtheorem{fact}[theorem]{事実}
\newtheorem{remark}[theorem]{remark}
\newtheorem*{remark*}{remark}
\def\proofname{証明}

%-------番号のカスタマイズ
\numberwithin{equation}{section}
\numberwithin{theorem}{section}

%-------マクロ
\newcommand{\groebner}[1]{Gr\"{o}bner#1}
\newcommand{\ideal}[1]{\langle #1 \rangle}
% \newcommand{\ht}{\mathrm{ht}}
% \newcommand{\hc}{\mathrm{hc}}
% \newcommand{\hm}{\mathrm{hm}}
\renewcommand{\algorithmicrequire}{\textbf{input:}}
\renewcommand{\algorithmicensure}{\textbf{output:}}
\newcommand{\ForTo}[1]{\; \mathbf{to} \; #1}
\newcommand{\ForToBy}[2]{\; \mathbf{to} \; #1 \; \mathbf{by} \; #2}
\newcommand{\ForEach}[1]{\mathbf{each} \; #1}
\algdef{SE}[DOWHILE]{Do}{doWhile}{\algorithmicdo}[1]{\algorithmicwhile\ #1} %do-while文
% ---使用例---
% \Do
% 	\State Something
% \doWhile{$i = j$}
% のように使う

