% chapter 03
\section{structural \groebner{} basis detection\cite{sturmfels1997structural}}
\groebner{} basis detectionの問題は,結局は項順序の同値類の数だけS多項式の計算をする必要があり,Buchbergerアルゴリズム等の前処理としての役割には適していない.そこで,Buchbergerの判定条件を用いて問題を簡略化したものが問題\ref{chapter03:problem:SGBD}のstructural \groebner{} basis detectionである.
\subsection{$n = k$のとき}
まず,変数の個数$n$と多項式集合の濃度$k$が等しいときを考える.後になってわかることだが,$n \ne k$のときも,この場合のアルゴリズムに帰着して考えることができる.
\par
次のような多項式を例に考える.
\begin{example}
	\label{chapter03:example:example_polynomial_set}
	多項式集合$F = \{f_1, \dots, f_n\}$を$n$変数の多項式環$K[\bar{X}]$の部分集合とし,
	$$f_i = X_{1}^{a_{i1}} + \cdots + X_n^{a_{in}} - 1$$
	と表されるとものとする.
\end{example}
このような多項式集合$F$に対して,項順序$\bm{w} \in \mathbb{R}_{+}^n$における各多項式の先頭項の集合は,
$$\mathrm{HT}_{\bm{w}}(F) = \{X_{\varphi(1)}^{a_{1\varphi(1)}}, \dots, X_{\varphi(i)}^{a_{i\varphi(i)}}\}$$
のように各多項式の先頭項のインデックスを表す写像$\varphi: \{1, \dots, n\} \to \{1, \dots, n\}$によって表すことができる.
SGBDの問題を解くためには,この$\mathrm{HT}_{\bm{w}}(F)$を適切に定める必要があるが,写像$\varphi$で表現可能な$n!$通りの中から探さなくてはならない.
\par
次の補題により,$\mathrm{HT}_{\bm{w}}(F)$は$n!$通りのうち$1$通りに絞られることがわかる.
\begin{lemma}[{\cite[Lemma5]{sturmfels1997structural}}]
	\label{chapter03:lemma:bipartite_graph}
	例\ref{chapter03:example:example_polynomial_set}の条件を満たす多項式集合$F$に対して,
	$$\prod_{i=1}^n a_{i\rho(i)} \ge \prod_{i=1}^n a_{i\varphi(i)}$$
	を満たすような$\rho$は存在しない.
\end{lemma}
\begin{proof}
	$f_i$の項$X_{\varphi(i)}^{a_{i\varphi(i)}}$を先頭項とするような項順序を$\bm{w} = (w_i)_{i \in \{1, \dots, n\}} \in \mathbb{R}_{+}^n$とし,それ以外の任意の像を$\rho \ne \varphi$とする.項順序$\bm{w}$の下で各多項式の先頭項は$X_{\varphi(i)}^{a_{i\varphi(i)}}$となるので,任意の$i \in \{1, 2, \dots, n\}$に対して
	$$w_{\varphi(i)}a_{i\varphi(i)} > w_{\rho(i)}a_{i\rho(i)}$$
	となる.つまり,$\varphi(i) \ne \rho(i)$を満たす$i$が少なくとも$1$つ存在する.
	従って,$n$個の不等式の総積によって,以下の式が得られる.
	\begin{align*}
		& \prod_{i=1}^n w_{\varphi(i)}a_{i\varphi(i)} > \prod_{i=1}^n w_{\rho(i)}a_{i\rho(i)} \\
		\Longleftrightarrow & \prod_{i=1}^n w_i a_{i\varphi(i)} > \prod_{i=1}^n w_i a_{i\rho(i)}
	\end{align*}
	両辺を$\displaystyle \prod_{i=1}^n w_i > 0$で割ると,
	$$\prod_{i=1}^n a_{i\rho(i)} < \prod_{i=1}^n a_{i\varphi(i)}$$
\end{proof}
この補題により,SGBDを解くために先頭項候補は,総積$\displaystyle \prod_{i=1}^n a_{i\varphi(i)}$
を最大化するような項の組
$$X_{\varphi(1)}^{a_{1\varphi(1)}}, \dots, X_{\varphi(n)}^{a_{n\varphi(n)}}$$
であることがわかる.このような項の組は二部グラフの最大マッチング問題を解くことによって求めることができる(詳細は後述のアルゴリズム\ref{chapter03:algorithm:SGBD}を参照).
\par
次に,実際にそのような項の組$X_{\varphi(1)}^{a_{1\varphi(1)}}, \dots, X_{\varphi(n)}^{a_{n\varphi(n)}}$がそれぞれの先頭項となるような項順序$\bm{w} \in \mathbb{R}_{+}^n$を求める方法を考える.以下の補題により,これは線形計画問題を解くことで求めることができることがわかる.
\begin{lemma}[{\cite[Lemma6]{sturmfels1997structural}}]
	\label{chapter03:lemmma:lin_prog}
	多項式集合$F$を例\ref{chapter03:example:example_polynomial_set}の条件を満たすような集合とし,$X^{\alpha_i}$を多項式$f_i \in F$の項とする.
	任意の$i \in \{1, \dots, n\}$と
	$X^{\beta_i} \ne X^{\alpha_i}$を満たす$f_i$の任意の項$X^{\beta_i}$に対して,指数ベクトルの差のベクトル$\alpha_i - \beta_i$を列として持つ行列を$\varGamma$とする.
	このとき,$\mathrm{HT}_{\bm{w}}(f_i) = X^{\alpha_i}$を満たすような項順序$\bm{w} \in \mathbb{R}_{+}^n$が唯一存在し,連立不等式$\varGamma \bm{w} > 0, \bm{w} > 0$を解くことによって求めることができる.
\end{lemma}
\begin{proof}
	$\mathrm{HT}_{\bm{w}}(f_i) = X^{\alpha_i}$という仮定から,行列$\varGamma$と$\bm{w}$との積は正となる必要がある.
\end{proof}

これらの補題をもとに,$n=k$のときにSGBDを解くアルゴリズムは以下のように記述することができる.
尚,SGBDの条件を満たすような項順序が存在しなかった場合には,ゼロベクトルを返すようにしている.
\begin{algorithm}[htbp]
	\label{chapter03:algorithm:SGBD}
	\caption{solving structural \groebner{} basis detection for $n = k$ {\cite[Algorighm7]{sturmfels1997structural}}}
	\begin{algorithmic}[1]
		\Require 多項式集合$F \subset K[\bar{X}]$
		\Ensure $F$が$\ideal{F}$の\groebner{}基底となるような項順序$\bm{w} \in \mathbb{R}_{+}^n$ or ゼロベクトル$\bm{0} \in \mathbb{R}^n$
		\For{$\ForEach{i, j \in \{1, 2, \dots, n\}$}}
			\If{$X_j^\alpha \notin T(f_i) \quad (\alpha \ne 0)$}
				\State $a_{ij} \gets 0$
			\EndIf
			\State $a_{ij} \gets (f_i\text{の単項式}X_j\text{の最大指数})$
			\If{$X_j^{a_{ij}}X^\alpha \in T(f_i) \quad (\alpha \ne 0)$}
				\State $a_{ij} \gets 0$
			\EndIf
		\EndFor
		\State 頂点が$2n$個ある二部グラフ$B = \{\bar{V}, \bar{E}\}$を次のように構成
		$$\bar{V} = \{u_1, \dots, u_n, v_1, \dots, v_n\}, \quad (u_i, v_i) \in \bar{E} \quad(a_{ij} > 0 \text{のときのみ})$$
		\State $B$の最大マッチング$M$を求める.
		\If{$\left| M \right| < n$}
			\State \Return $\bm{0} \in \mathbb{R}^n$
		\Else 
			\State $M = \{(u_i, v_{\varphi(i)})\;:\; i \in \{1, 2, \dots, n\}\}$
		\EndIf
		\State 行列$\varGamma$を補題\ref{chapter03:lemmma:lin_prog}と同じように構成し,
		$$\begin{cases}
			\varGamma \bm{w} > 0\\
			\bm{w} > 0
		\end{cases}$$
		を解く.
		\State \Return $\bm{w}$
	\end{algorithmic}
\end{algorithm}

\begin{theorem}
	アルゴリズム 2は正当性と有限停止性を有する.
\end{theorem}
\begin{proof}
	まず,1行目のfor文において,入力の多項式を例\ref{chapter03:example:example_polynomial_set}の条件を満たすような項のみを残すように変換している.10行目〜16行目は補題\ref{chapter03:lemma:bipartite_graph}に基づいて二部グラフを構成し,17行目では補題\ref{chapter03:lemmma:lin_prog}に基づいて行列を構成し,線形計画問題を解いている.
\end{proof}
このアルゴリズムでは,大きく分けて
\begin{enumerate}
	\item 単項のみから成り,且つ倍単項式が同じ多項式に存在しないような項のみを残す
	\item 二部グラフの最大マッチング問題を解く
	\item 線形計画問題を解く
\end{enumerate}
という$3$つのステップがある.
2つ目のステップでは,Hungarian method\cite{plummer1986matching}にて,3つ目のステップではKhachian's Ellipsoid method\cite{schrijver1998theory}などの方法を採用することで,アルゴリズム全体は多項式時間で解くことができる\cite{sturmfels1997structural}.

\subsection{$n \ne k$のとき}
次に,$n \ne k$のときを考える.$n < k$を満たすときには,系\ref{chapter02:cororally:2}で示されている通り,SGBDで求めるべき互いに素な項が存在しない.そのため,$n > k$のときを考える.
\begin{lemma}[{\cite[Lemma9]{sturmfels1997structural}}]
	$\bar{X} \cap \bar{Y} = \phi$を満たすような新たな変数の集合として$\bar{Y} = \{y_1, \dots, y_p\}$を定義する.
	$S_1 \cup \cdots \cup S_p$を変数$\{ x_1, \dots, x_n \}$の分割とし,写像$\pi$を$x_i \mapsto y_i$で定義される$K[\bar{X}]$から$K[\bar{Y}]$への写像とする.
写像$\varphi:\{1, \dots, p\} \to \{1, \dots, p\}$を今までと同様のものとする.
$\mathrm{HT}_{\bm{w}_y}(\pi(f_i))$が$y_{\varphi(i)}$のべき乗であるような$K[\bar{Y}]$上の項順序$\bm{w}_y$がある場合に限り,$\mathrm{HT}_{\bm{w}_x}(f_i)$が$S_{\varphi(i)}$にサポートを持つような$K[\bar{X}]$上の項順序$\bm{w}_x$が存在する.
\end{lemma}
\begin{proof}
	項順序$\bm{w}_y$を,$\displaystyle \bm{w}_y(j) = \sum_{i \in S_j}\bm{w}_x(i)$を満たすようにとることで,
	$$\pi(X^\alpha) \prec_{\bm{w}_y} \pi(X^\beta) \Longrightarrow X^\alpha \prec_{\bm{w}_x} X^\beta$$
	を満たすことができる.
\end{proof}
この補題のように$n = k$となるように変数の組み合わせを構成することで,$n = k$のときのアルゴリズム(アルゴリズム 2)に帰着できることがわかった.



















