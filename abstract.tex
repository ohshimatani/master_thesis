\phantom{fight!}\par

計算機代数の分野において重要な計算の$1$つに\groebner{}基底の計算がある.
\groebner{}基底の計算はBuchbergerアルゴリズム\cite{buchberger2006bruno}に代表されるように,イデアルと項順序を指定して,一定の条件を満たすようなイデアルの基底を求めるものである.
$F_4$アルゴリズム\cite{faugere1999new}など他の多くのアルゴリズムでも,Buchbergerアルゴリズムの系譜が受け継がれており,基本的なアルゴリズムの考え方は変わっていない.
\par
一方で,以下の例のように,与えられた多項式集合$F$がある項順序においてそのままイデアル$\ideal{F}$の\groebner{}基底となっているような場合が存在する.
\begin{example*}
	以下の多項式集合$F$は,$z \succ y \succ x$の全次数辞書式順序及び全次数逆辞書式順序においてイデアル$I = \ideal{F}$の\groebner{}基底となっている.
	$$F = \left\{ 2xy + yz, \; x^2 + y + z \right\} \subset \mathbb{C}[x, y, z]$$
\end{example*}
このように,直接\groebner{}基底の計算をする前に,計算対象である$F$自身がそのまま\groebner{}基底となっているような項順序を求めることができれば,項順序を指定しない\groebner{}基底が必要な場面では有効な計算方法であることが考えられる.また,仮に一定の項順序が必要な場面でも,FGLMアルゴリズム\cite{faugere1993efficient}や\groebner{} walk\cite{collart1993grobner}などに代表されるchange of orderingのアルゴリズムを活用することによって,有用な計算手段となることが考えられる.
\par
例えば,連立代数方程式の求解のための計算では,多くの場合で辞書式順序(一般的には消去順序)での\groebner{}基底が必要となるが,辞書式順序での計算は遅くなることが知られており,入力の多項式集合の大きさによっては
\par
このように「そのまま\groebner{}基底である」ような項順序を検出する問題は,Sturmfelsらによって解かれた既知の問題であり,\emph{\groebner{} basis detection}\cite{gritzmann1993minkowski}や\emph{Structural \groebner{} basis detection}\cite{sturmfels1997structural}という名前が付けられている.
本論文では,これらの問題において,問題の設定をパラメータを伴った多項式環へと拡張し,それに付随して発見された定理についても取り上げる.
まず第\ref{chapter01:chapter_num}章では,Sturmfelsらとは違ったアプローチから,\emph{Structural \groebner{} basis detection}の問題を捉え,そこで新たに発見された定理を紹介する.
次に,第\ref{chapter02:chapter_num}章では,\emph{\groebner{} basis detection}と\emph{Structural \groebner{} basis detection}について,既に知られている部分について述べる.
第\ref{chapter03:chapter_num}章では,これらの問題をパラメータを伴った多項式環へと拡張するために,パラメータ空間を分割するためのアルゴリズムの直接的な方法を紹介する.
最後に,第\ref{chapter04:chapter_num}章では,パラメータ空間の分割を効率化するための議論を行い,最終的なアルゴリズムを完成させる.
