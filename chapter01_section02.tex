% chapter01_はじめに
\section{基本的な定義}
以下では本論文全体を通して使用される記法を定義しておく.
\par
$\mathbb{N}$は$0$以上の整数全体の集合とする.$K$を体とし,$L$を$K$の代数閉包とする.
$n$個の変数全体の集合を$\bar{X} = \{x_1, \dots, x_n\}$とし,$m$個のパラメータ全体の集合を$\bar{A} = \{a_1, \dots, a_m\}$とする(ただし,$\bar{X}\cap \bar{A} = \phi$).
このような集合$\bar{X}, \bar{A}$に対して,多項式環$K[\bar{X}, \bar{A}]$を,主変数$\bar{X}$に関する多項式環$(K[\bar{A}])[\bar{X}]$とみなす.
\par
多項式環上のイデアルを多項式集合$F$を用いて$\ideal{F}$と表す.イデアル$I$に対して,その根基イデアルを$\sqrt{I}$と表す.
\par
多項式$f \in K[\bar{X}]$に対して$T_{\bar{X}}(f)$を$\bar{X}$に関する$f$の$\bar{X}$に関する$0$でない項全体の集合,$M_{\bar{X}}(f)$を$f$の$\bar{X}$に関する$0$でない単項式全体の集合とする.多項式集合$F$に対しては,
$$\displaystyle T_{\bar{X}}(F) = \bigotimes_{i=1}^k T_{\bar{X}}(f_i), \quad M_{\bar{X}}(F) = \bigotimes_{i=1}^k M_{\bar{X}}(f_i)$$
のように定義する.
単項式$m \in K[\bar{X}]$に対して,$\mathrm{coeff}_{\bar{X}}(m)$を$m$の$\bar{X}$に関する係数,$\mathrm{term}_{\bar{X}}(m)$を$m$の$\bar{X}$に関する項として定義する.
尚,多項式環の主変数が明らかな場合には単に$T(f)$や$\mathrm{coeff}(m)$のように省略して書くこともある.
$n$変数の項全体の集合を$T_n = \left\{ x_1^{e_1} \cdots x_n^{e_n} : e_i \in \mathbb{N} \right\}$とする.
\par
項順序を次のように定義する.
\begin{definition}[項順序]
	$T_n$における全順序$\prec$が項順序であるとは,
	\begin{itemize}
		\item 任意の$t \in T_n$に対し$1 \preceq t$
		\item 任意の$t_1, t_2, s \in T_n$に対し,$t_1 \preceq t_2 \Longrightarrow s\cdot t_1 \preceq s\cdot t_2$
	\end{itemize}
	を満たすことを言う.
\end{definition}
項順序$\prec$において,多項式$f\in K[\bar{X}]$に含まれる項で,最も項順序が大きい単項式を$\mathrm{hm}_{\prec}(f)$とし,$f$の\textbf{頭単項式}と呼ぶ.
また,
$\mathrm{ht}_{\prec}(f) = \mathrm{term}_{\bar{X}}(\mathrm{hm}_{\prec}(f)), \; \mathrm{hc}_{\prec}(f) = \mathrm{coeff}_{\bar{X}}(\mathrm{hm}_{\prec}(f))$と定義し,それぞれ\textbf{頭項,頭係数}と呼ぶ.
項順序が明らかな場合には,$\mathrm{hm}_{\prec}(f)$を省略して単に$\mathrm{hm}(f)$などと書くこともある.
また,多項式集合$F$に関して,$\mathrm{HM}_{\prec}(F) = \{ \mathrm{hm}_{\prec}(f_i) \;:\; f_i \in F \}$と定義する($\mathrm{HT}_{\prec}(F)\text{と}\mathrm{HC}_{\prec}(F)$についても同様に定義する).
$K[\bar{X}]$の$0$でないイデアル$I$に対しては,$\mathrm{HT}_\prec(I)$を$I$の元の頭項全体の集合として定義する($\mathrm{HT}_{\prec}(I)\text{と}\mathrm{HC}_{\prec}(I)$についても同様に定義する).
\par
項$t \in T_n$の指数ベクトルを$e(t) \in \mathbb{N}^{n}$と表す.
重み行列$M$を用いて表されるmatrix order $\prec$を以下のように定義する.
\begin{definition}[matrix order]
	項順序$\prec$が重み行列$M \in \mathbb{R}^{d\times n}$を用いて表されるmatrix orderであるとは,項$t_1, t_2 \in T_n$の指数ベクトル$e(t_1), e(t_2) \in \mathbb{N}^{n}$に対して,
	$$t_1 \prec t_2 \Longleftrightarrow Me(t_1) <_{\ne} Me(t_2)$$
	を満たすことを言う.ただし,$<_{\ne}$や$>_{\ne}$は,ベクトルの等しくない最初の成分での比較を表す不等号である.$d=1$のときは,通常の大小関係での比較となる.
\end{definition}
matrix orderは任意の項順序を表現可能である\cite{MR826583}ということがわかっており,column full rankな行列を考えれば十分であるということもわかっている.
\par
項順序を$\prec$で固定する.多項式$f, g \in K[\bar{X}]$に対し,$f$に含まれる単項式$m$が$\mathrm{ht}(g)$で割り切られるとする.このとき,$h = f - \frac{m}{\mathrm{hm}(g)}g$に対し,$f \to_g h$と書き,\textbf{$f$の$g$での単項簡約}と呼ぶ.この操作を$0$回を含む有限回繰り返し,これ以上単項簡約できない$h$が得られたとき,$h$を\textbf{$f$の$g$による正規形(normal form)}と呼び,$h = \mathrm{nf}_g(f)$で表す.
また,有限な多項式集合$G=\left\{g_i :i \in \{1,2, \dots\}\right\} \subset K[\bar{X}]$において,$G$に含まれる多項式$g_i$で$f$を単項簡約することで$h$が得られるとき,同様に$f \to_G h$と書く.
また,$g_i$による単項簡約を$0$回を含む有限回繰り返すことで,これ以上$g_i$による単項簡約ができない$h$が得られるとき,$h$を\textbf{$f$の$G$による正規形}と呼び,$h = \mathrm{nf}_G(f)$で表す.
\begin{definition}[S多項式]
	項順序を$\prec$で固定し$f, g \in K[\bar{X}]$とする.このとき,$f, g$の\textbf{S多項式}を次のように定義する.
	$$\mathrm{Spoly}(f, g)=\frac{\mathrm{lcm}(\mathrm{ht}(f), \mathrm{ht}(g))}{\mathrm{hm}(f)}\cdot f - \frac{\mathrm{lcm}(\mathrm{ht}(f), \mathrm{ht}(g))}{\mathrm{hm}(g)}\cdot g$$
\end{definition}
これらを用いて,\groebner{}基底を次のように定義する.
\begin{definition}[\groebner{}基底]
	多項式集合$F \subset K[\bar{X}]$と$F$が生成するイデアル$I = \ideal{F}$に対して,
	$$\ideal{\mathrm{HT}_\prec(I)} = \ideal{\mathrm{HT}_\prec(f_1), \dots, \mathrm{HT}_\prec(f_k)}$$
	が満たされるとき,$F$を項順序$\prec$に対する\textbf{\groebner{}基底}であるという.
\end{definition}
パラメータ$\bar{A}$を含む$\bar{X}$に関する多項式環$K[\bar{X}, \bar{A}]$における包括的\groebner{}基底系及び包括的\groebner{}基底を次のように定義する.
\nocite{鈴木晃2007グレブナー基底を用いた包括的グレブナー基底計算}
\begin{definition}[包括的\groebner{}基底系(Comprehensive \groebner{} system; CGS)]
	主変数$\bar{X}$に関する項順序を$\prec$で固定する.
	多項式集合$F \subset K[\bar{X}, \bar{A}]$と$L^m$の代数構成的集合$S$に対して,
	$$\mathcal{G} = \{(S_1, G_1), \dots, (S_1, G_\ell)\}$$
	が$F$の$S$上の\textbf{包括的\groebner{}基底系(comprehensive \groebner{} system; CGS)}であるとは,
	\begin{itemize}
		\item $S_1, \dots, S_\ell$は$L^m$の構成的部分集合
		\item $G_1, \dots, G_\ell$は$K[\bar{X}, \bar{A}]$の有限な部分集合
		\item $\displaystyle \bigcup_{i=1}^{\ell} S_i \supseteq S, \; S_i \cap S_j = \phi \quad (\forall i,j \in \{ 1, \dots, \ell \}, \; i \ne j)$
		\item 任意の$\bar{a} \in S_i$に対して,$\sigma_{\bar{a}}(G_i)$が$\sigma_{\bar{a}}(F)$の$L[\bar{X}]$における\groebner{}基底である
	\end{itemize}
	を満たすときにいう.特に$S = L^m$のとき,$\mathcal{G}$を単に$F$の包括的\groebner{}基底系であるという.
\end{definition}

\begin{definition}[包括的\groebner{}基底(comprehensive \groebner{} basis; CGB)]
	多項式集合$F,\; G \subset K[\bar{X}, \bar{A}]$に対して,$\{(L^m, G)\}$が$F$の包括的\groebner{}基底系であるとき,$G$を$F$の\textbf{包括的\groebner{}基底(comprehensive \groebner{} basis; CGB)}という.
\end{definition}






