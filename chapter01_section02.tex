% chapter01_はじめに
\section{基本的な定義}
以下では本論文全体を通して使用される記法を定義しておく.
\par
$\mathbb{N}$は$0$以上の整数全体の集合とする.$K$を体とし,$L$を$K$の代数閉包とする.
$n$個の変数全体の集合を$\bar{X} = \{x_1, \dots, x_n\}$とし,$m$個のパラメータ全体の集合を$\bar{A} = \{a_1, \dots, a_m\}$とする.ただし,$\bar{X}\cap \bar{A} = \phi$.$T(f)$を$f$の$0$でない項全体の集合とする.
$n$変数の項全体の集合を$T_n = \left\{ x_1^{e_1} \cdots x_n^{e_n} : e_i \in \mathbb{N} \right\}$とする.
項順序を次のように定義する.
\begin{definition}[項順序]
	$T_n$における全順序$\prec$が項順序であるとは,
	\begin{itemize}
		\item 任意の$t \in T_n$に対し$1 \prec t$
		\item 任意の$t_1, t_2, s \in T_n$に対し,$t_1 \prec t_2 \Longrightarrow s\cdot t_1 \prec s\cdot t_2$
	\end{itemize}
	を満たすことを言う.
\end{definition}
項順序$\prec$において,多項式$f\in K[\bar{X}]$に含まれる項で,最も項順序が大きい単項式を$\mathrm{hm}_{\prec}(f)$,その係数を除いた部分を$\mathrm{ht}_{\prec}(f)$,その係数を$\mathrm{hc}_{\prec}(f)$と定義し,それぞれ\textbf{頭単項式,頭項,頭係数}と呼ぶ.
項順序が明らかな場合には,$\mathrm{hm}_{\prec}(f)$を単に$\mathrm{hm}(f)$などと書くこともある.
また,多項式集合$F$に関して,$\mathrm{HM}_{\prec}(F) = \{ \mathrm{HM}_{\prec}(f_i) \;:\; f_i \in F \}$と定義する($\mathrm{HT}_{\prec}(F), \mathrm{HC}_{\prec}(F)$についても同様に定義する).
項$t \in T_n$の指数ベクトルを$e(t) \in \mathbb{N}^{n}$と表す.
重み行列$M$で表されるmatrix order$\prec_M$を以下のように定義する.
\begin{definition}[matrix order]
	項$t_1, t_2 \in T_n$の指数ベクトル$e(t_1), e(t_2) \in \mathbb{N}^{n}$に対し,行列$M \in \mathbb{R}^{d\times n}$がmatrix orderであるとは,
	$$t_1 \prec_M t_2 \Longleftrightarrow Me(t_1) <_{\ne} Me(t_2)$$
	を満たすことを言う.ただし,$<_{\ne}$や$>_{\ne}$は,ベクトルの等しくない最初の成分での比較を表す不等号である.$d=1$のときは,通常の大小関係での比較となる.
\end{definition}
matrix orderは任意の項順序を表現可能である\cite{MR826583}ということがわかっており,column full rankな行列を考えれば十分であるということもわかっている.
\par
項順序を$M$とする.多項式$f, g \in K[\bar{X}]$に対し,$f$に含まれる単項式$t$が$\mathrm{ht}_M(g)$で割り切られるとする.このとき,$h = f - \frac{t}{\mathrm{ht}_M(g)}g$に対し,$f \to_g h$と書き,\textbf{$f$の$g$での単項簡約}と呼ぶ.この操作を$0$回を含む有限回繰り返し,これ以上単項簡約できない$h$が得られたとき,$h$を\textbf{$f$の$g$による正規形(normal form)}と呼び,$h = \mathrm{nf}_g(f)$で表す.
また,有限な多項式集合$G=\left\{g_i :i \in \{1,2, \dots\}\right\} \subset K[\bar{X}]$において,$g_i \in G$で$f$を単項簡約することを繰り返すことで$h$が得られるとき,同様に$h$を\textbf{$f$の$G$による正規形}と呼び,$h = \mathrm{nf}_G(f)$で表す.
\begin{definition}[S多項式]
	項順序を$M$とし$f, g \in K[\bar{X}]$とする.このとき,$f, g$のS多項式を次のように定義する.
	$$\mathrm{Spoly}(f, g)=\frac{\mathrm{hc}_M(g)\cdot \mathrm{lcm}(\mathrm{ht}_M(f), \mathrm{ht}_M(g))}{\mathrm{ht}_M(f)}\cdot f - \frac{\mathrm{hc}_M(f)\cdot \mathrm{lcm}(\mathrm{ht}_M(f), \mathrm{ht}_M(g))}{\mathrm{ht}_M(g)}\cdot g$$
	\end{definition}








