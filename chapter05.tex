% chapter 05
本研究では,GBD及びSGBDについての議論を,特にSGBDについて,パラメータを伴った場合への拡張を行った.SGBDの計算は,二部グラフの計算と線形計画法に帰着されるため,最初にパラメータ空間を包括的に分割できれば,新たな場合分けが発生することはない.パラメータ空間の分割の効率化に関しても,affine Newton polyhedronを用いた方法や,多項式環の加群のCGSの計算により行えることもわかった.
\par
一方で,GBDのパラメータを伴った場合への拡張はSGBDのように単純にはいかない.通常のGBDの計算におけるS多項式同士の計算において,パラメータ空間の新たな場合分けが発生するためである.そのため,CGSを求めるSuzuki-Satoアルゴリズム\cite{suzuki2003alternative}のように場合分けをしながらアルゴリズムを進めてゆく方法が考えられるが,その効率的な方法については今後の課題である.
\par
また,そもそも多項式集合がパラメータを伴っているか否かに関わらず,SGBDの計算で,多項式集合がそのまま\groebner{}基底となっているような項順序が検出されるとは限らず,多くの場合では項順序の検出がなされない.それゆえ,逆に項順序が検出されやすいような条件(対象の多項式環や多項式の性質など)を発見することも今後の課題である.