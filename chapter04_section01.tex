% chapter 04
\section{はじめに}
第\ref{chapter03:chapter_num}章では,\groebner{} basis detection及びstructural \groebner{} basis detectionにおけるSturmfelsらの先行研究について述べた.
この章では,特にstructural \groebner{} basis detectionの問題において,対象の多項式環が,パラメータを伴った多項式環へと拡張された場合について述べる.
\par
簡単な例を考える.
\begin{example}
	\label{chapter04:example:PD_example}
	$K[a][x, y, z]$を主変数$x, y, z$に関する多項式環とする.次のような多項式集合$F\subset K[a][x, y, z]$を考える.
	\begin{equation*}
		F = \left\{
			\begin{aligned}
				f_1 &= x + (a-3)y^2, \\
				f_2 &= x^3 + z, \\
				f_3 &= y + z^3
			\end{aligned}
		\right\}
	\end{equation*}
	\begin{itemize}
		\item $a - 3 \ne 0$のとき\\
			$f_1$の$y^2$の項は$0$にならず,SGBDを行うと,$$\mathrm{HT}_{\bm{w_1}}(f_1) = y^2, \quad \mathrm{HT}_{\bm{w_1}}(f_2) = x^3,\quad \mathrm{HT}_{\bm{w_1}}(f_3) = z^3$$を満たすような項順序$\bm{w_1}$を求めたい.実際,$\bm{w_1}=(1, 2, 3)$などが条件を満たすベクトルとなる.
		\item $a - 3 = 0$のとき\\
			$f_1$の$y^2$の項が$0$となるため,残った項でSGBDを行うと,$$\mathrm{HT}_{\bm{w_2}}(f_1) = x,\quad \mathrm{HT}_{\bm{w_2}}(f_2) = z,\quad \mathrm{HT}_{\bm{w_3}}(f_3) = y$$を満たすような項順序$\bm{w_2}$を求めたい.実際,$\bm{w_2} = (1, 16, 4)$などが条件を満たすベクトルとなる.
	\end{itemize}
\end{example}
このように,パラメータを伴った多項式環においては,パラメータの付いた係数が$0$となるか否かによって多項式に含まれる項が変わるため,項を確定させるためにパラメータ空間を包括的に分割する必要がある.逆に,最初にパラメータ空間の分割を行い項を確定させる事ができれば,SGBDにおいてはその後に項同士の計算が発生しないため,パラメータを伴わない通常のSGBDに帰着することができる.
このように,パラメータを伴った多項式環におけるSGBDの問題を,次のように定義する.尚,以降は,多項式環$K[\bar{X}, \bar{A}]$を,主変数$\bar{X}$に関する多項式環$(K[\bar{A}])[\bar{X}]$と同じものとして考える.
\newpage
\begin{problem}[comprehensive structural \groebner{} basis detection; comprehensive SGBD]
	$S \subseteq L^m$を代数構成的集合(algebraically constructible subsets)とする.多項式集合$F \subset K[\bar{X}, \bar{A}]$に対して,
	集合$\tilde{\mathcal{G}}$を次の条件を満たすように構成せよ.
	\begin{itemize}
		\item $S_1, \dots, S_\ell$は構成的部分集合
		\item $\displaystyle \bigcup_{i=1}^\ell S_i \supseteq S, \; S_i \cap S_j = \phi \quad (\forall i, j \in \{1, \dots, \ell\}, \; i \ne j)$
		\item $\tilde{\mathcal{G}} = \{(S_1, \bm{w_1}), \dots, (S_\ell, \bm{w_\ell}))\}$
		\item $\bar{a} \in S_i$に対し,$\sigma_{\bar{a}}(F)$が項順序$\bm{w_i}$におけるイデアル$\ideal{\sigma_{\bar{a}}(F)}$の\groebner{}基底
		\item そのような項順序がない場合,$\bm{w_i} = 0$
	\end{itemize}
\end{problem}











