% chpter 02
\section{項順序$M$の導出(Step2)}
求めたい項順序$M$を,
$$
	M = \begin{pmatrix}
		\bm{m_1} \\
		\vdots \\
		\bm{m_n}
	\end{pmatrix}
	=
	\begin{pmatrix}
		m_{11} & \cdots & m_{1n} \\
		\vdots & \ddots & \vdots \\
		m_{n1} & \cdots & m_{nn}
	\end{pmatrix}
$$
とする.
matrix orderの定義より,項$t_1, t_2$が$t_1 \succ_M t_2$を満たすとき,$Me(t_1) >_{\ne} Me(t_2)$が満たされる.これを一般の等号と不等号を用いて表すと以下のようになる.
\begin{equation}
	\left\{
	\begin{aligned}
		\bm{m_1} \cdot e(t_1) &= \bm{m_1} \cdot e(t_2)\\
		&\;\; \vdots \\
		\bm{m_{\ell-1}} \cdot e(t_1) &= \bm{m_{\ell-1}} \cdot e(t_2)
	\end{aligned}
	\right.
	\quad
	,
	\quad \bm{m_{\ell}} \cdot e(t_1)> \bm{m_{\ell}} \cdot e(t_2) \notag
\end{equation}
ここで,$\ell$は,$M$の第$(\ell-1)$行ベクトルまでとの積が等しく,第$\ell$ベクトルで初めて差がつくときことを表している.
この連立不等式を解くことによって,重み行列$M$を求めることができるが,その効率的な方法については検討中である.