
% \title{パラメータを伴った\groebner{}基底の構造的な検出について}
% \subtitle{SGBD}
% \author{大島谷遼}
% \date{\today}

% ------title
\begin{center}
	{
		\fontsize{15truept}{0truept}\selectfont
		パラメータを伴った\groebner{}基底の構造的な検出について
	}
\end{center}

% ------subtitle
\begin{flushright}
	--- Comprehensive structural \groebner{} basis detection ---
\end{flushright}

% ------other
\begin{minipage}{0.49\columnwidth}
	\phantom{fight!}
\end{minipage}
\begin{minipage}{0.49\columnwidth}
	{
		\fontsize{12truept}{17truept}\selectfont
		\begin{tabular}[htbp]{ll}
			所属専攻 &人間環境学専攻\\
			学籍番号 &208D418D\\
			学\hspace{7.33truept}生\hspace{7.33truept}氏\hspace{7.33truept}名 &大島谷 遼\\
			指導教員氏名 &長坂 耕作  准教授
		\end{tabular}
	}
\end{minipage}
\\

% ------main
\groebner{}基底とは,1965年にBruno Buchbergerによって発表された単項簡約(多変数多項式の割り算のようなもの)を一意に行うことのできるイデアルの生成系である.\groebner{}基底を計算することによって連立代数方程式の厳密解を求めたりすることできるため,計算機代数分野において重要な研究対象となっている.他にも,代数統計学・代数幾何学・組合せ論などの他分野への応用も盛んに行われている.
また,昨今の計算機のめざましい発展により,機械学習をはじめとするビッグデータを扱った数値計算が学術領域に限らず世界的に行われているが,近似解を求める数値計算に対し,厳密解を求める\groebner{}基底の計算が必要な場合も多く存在し,特にMathematicaやMapleなどに代表される数式処理システムの内部算法として欠かせないものとなっている.
\par
本論文では,パラメータを伴った多項式集合$F$がそのまま$\ideal{F}$の\groebner{}基底となっているような項順序を求める方法を記す.
パラメータを伴っていない多項式集合において,この問題は\groebner{} basis detection(GBD)と呼ばれている.GBDはSturmfelsらにより,affine Newton polyhedronを用いて項順序を同値類に分類し,それぞれの代表元においてS多項式が全て$0$へと簡約される($\Leftrightarrow$ \groebner{}基底となっている)ことを確かめるという方法が提案されている\cite{gritzmann1993minkowski}.しかし,このアルゴリズムでは結局S多項式の簡約化の計算を(多くの場合では)複数回行わなくてはならない.また,GBDの問題のクラスはNP-hardであることも分かっている\cite[Corollary11]{sturmfels1997structural}.そこで,問題の条件を,「先頭項同士が全て互いに素となるような項順序は存在するか」(Buchbergerの判定条件)という\groebner{}基底となるための十分条件に置き換えて問題を簡単にする.この問題をstructural \groebner{} basis detection(SGBD)と呼ぶ.SGBDについても,Sturmfelsらによって二部グラフの最大マッチング問題と線形計画問題に帰着されたアルゴリズムが与えられている.また,SGBDの問題のクラスはNP-complete\cite[Theorem2]{sturmfels1997structural},特に多項式集合の濃度と変数の数の差を固定したときの問題のクラスはPである\cite[Corollary3]{sturmfels1997structural}ことが証明されている.
\par
本論文では,特にこのSGBDに着目し,多項式集合がパラメータを伴った場合についての拡張を行う.パラメータを伴ったSGBDは,最初にパラメータ空間を包括的に分割することで,後に新たな場合分けを発生させずに通常のSGBDのアルゴリズムを計算することができる.また,このような直接的な方法の他にも,affine Newton polyhedronや,多項式環の加群の包括的\groebner{}基底系を用いてパラメータを伴ったSGBDを効率化する方法についても提示している.








