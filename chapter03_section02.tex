% chapter 03
\section{\groebner{} basis detection}
\groebner{} basis detectionの問題を解くあたって,多項式集合の項順序に幾何的な解釈を与えることで,項順序の同値類に分けて考えることができるようになる.ここで,多項式集合$F$に対して,項順序$\bm{w_1}, \bm{w_2}$は,$\mathrm{HT}_{\bm{w_1}}(F) = \mathrm{HT}_{\bm{w_2}}(F)$を満たすときに同値であるという.まずは前提とする定義を与える.
\subsection{前提とする定義}

\begin{definition}[{\cite[p8,9, Definition3.1]{freeke2009linking}}]
	集合$\mathcal{U, V} \subseteq \mathbb{R}^d$に対して,
	\begin{itemize}
		\item $\mathcal{U}$が\emph{凸(convex)}であるとは,$$\forall \bm{u, v} \in \mathcal{U}, \; \lambda \in \mathbb{R}, \; 0 \le \lambda \le 1, \; \lambda\bm{u} + (1 - \lambda) \bm{v} \in \mathcal{U}$$を満たすときをいう.
		\item \emph{凸多面体(convex polyhedron)}とは,有限個の半空間の共通部分として得られる凸型の集合である.
		\item 集合$\mathcal{V}$の\emph{凸包(convex hull)}とは,その集合を含む$\mathbb{R}^d$のすべての凸部分集合の共通部分と定義される.
		\item  $\mathcal{U}$は有界な多面体である場合,\emph{超多面体(polytope)}と呼ばれる.すべてのpolytopeは,有限の点の集合のconvex hullである.
		% また,polytope $P \subset \mathbb{R}^n$の$\mathcal{V}$-presentationとは,$d$個の点$v_1, \dots, v_d$によって$P = \mathrm{conv} \{v_1, \dots, v_d\}$と表されるときをいう.ただし,$1 \le n < d$.
		% \item 凸多面体$P$の\emph{面(face)}とは,$P$とそれに接する任意の超平面との交点である.$P$のすべての面は,$\mathbb{R}^d$の任意のベクトル$\bm{c}$に対して$\mathrm{face}_c(P) = \left\{ \bm{x} \in P \; : \; \bm{cx} \ge \bm{cy}, \; \forall \bm{y} \in P \right\}$の部分集合である.
		% \item 多面体$P$の面$F$の\emph{次元(demension)}は$\bm{v} \in F$としたときに$\mathbb{R}^d$の部分空間$\bm{v} + H$の次元として定義される.また,$H$はベクトル$\bm{u} \in F, \; \bm{u - v}$によってspanされ,$\bm{v} + H$を$F$の\emph{affine span}と呼ぶ.
		% \item $d$-次元polytopeの$(d - 1)$-次元面を\emph{切子面(facets)}と呼ぶ.
		\item $\mathbb{R}^d$の凸多面体(convex polyhedron)の\emph{円錐(cone)}$C$は,$$\forall \bm{u, v} \in C, \; \lambda \in \mathbb{R}_{\ge 0}, \; \bm{u+v}, \lambda\bm{u} \in c$$のように定義される.
	\end{itemize}
\end{definition}
$2$つ以上のpolytopeの和を次のように定義する.
\begin{definition}[Minkowski和, {\cite[p247]{gritzmann1993minkowski}}]
	$2$つのpolytope $P_1, P_2 \subset \mathbb{R}^n$に対して,\emph{Minkowski和}$P_1 + P_2$を
	$$P_1 + P_2 = \{ x \in \mathbb{R}^n \;:\; \exists x_1 \in P_1, \exists x_2 \in P_2, x = x_1 + x_2\}$$
	のように定義する.
	Minkowski和は,可換であり結合法則が成り立つため,$2$つ以上のpolytopeにも自然に一般化することができる.
\end{definition}
変数の集合$\bar{X}$に対し,$\bar{X}$の任意の指数ベクトル$\alpha \in \mathbb{N}^n$とし,$K[\bar{X}]$の任意の単項式を$X^\alpha$と表現する.また,ベクトル$\bm{u}, \bm{v}$の内積を$(\bm{u}, \bm{v})$と表す.集合$\mathbb{R}_{-}$は$0$以下の実数全体の集合である.
\begin{definition}[Newton polytope]
	多項式$\displaystyle f = \sum_{i=1}^t c_i X^{\alpha_i}$の\emph{Newton polytope} $\mathcal{N}(f)$を,$\mathbb{R}^n$における単項式のconvex hullで定義する.つまり,
	$$\mathcal{N}(f) = \mathrm{conv}\{\alpha_1, \dots, \alpha_t\}$$
	である.
	また,多項式集合$F = \{f_1, \dots, f_k\} \subset K[\bar{X}]$のNewton polytopeを,それぞれの多項式のNewton polytopeのMinkowski和
	$$\mathcal{N}(F) = \mathcal{N}(f_1) + \cdots + \mathcal{N}(F)$$
	で定義する.
	また,多項式$f$の\emph{affine Newton polyhedron} $\mathcal{N}_{\mathrm{aff}}(f)$をMinkowski和
	$$\mathcal{N}_{\mathrm{aff}}(f) = \mathcal{N}(f) + \mathbb{R}_{-}^n$$
	で定義する.
	多項式集合$F$についても同様に,$$\mathcal{N}_{\mathrm{aff}}(F) = \mathcal{N}(F) + \mathbb{R}_{-}^n$$で定義する.
\end{definition}

\begin{remark}
	多項式のNewton polytopeのMinkowski和は,多項式の積のMinkowski和と対応している.つまり,
	多項式$f_1, \dots, f_k \subset K[\bar{X}]$に対して,
	$$\mathcal{N}(f_1) + \mathcal{N}(f_2) + \cdots + \mathcal{N}(f_k) = \mathcal{N}(f_1 \cdots f_k)$$
	である.
\end{remark}

ここで,例として多項式$f = x^3y^2 + xy^3 + xy \subset K[x, y]$のNewton polytopeとaffine Newton polyhedronを図示する.
\\

\begin{figure}[htbp]
	\begin{minipage}[h]{0.49\columnwidth}
		\centering
		\begin{tikzpicture}
			\label{chapter03:figure:NewtonPolytopeExample}
			% x軸y軸の描画
			\draw[->,>=stealth,semithick] (-1,0)--(4,0)node[above]{$x$};
			\draw[->,>=stealth,semithick] (0,-1)--(0,4)node[right]{$y$};
			% 点の定義
			\coordinate (A) at (1, 1);
			\coordinate (B) at (1, 3);
			\coordinate (C) at (3, 2);
			\coordinate (internal_point) at (1.5, 2);
			\coordinate (explanation_point) at (2, 3);
			% 点の描画
			\draw (0,0) node[below right]{$O$};
			\draw (A) node [below, circle] {$(1, 1)$};
			\draw (B) node [above, circle] {$(1, 3)$};
			\draw (C) node [right, circle] {$(3, 2)$};
			\draw (explanation_point) node [above right] {$\mathcal{N}(f)$};
			\fill (A) circle (2pt) (B) circle (2pt) (C) circle (2pt);
			% 先の描画
			\draw (A)--(B);
			\draw (B)--(C);
			\draw (C)--(A);
			\draw [<-] (internal_point)--(explanation_point);
			% 斜線の描画
			\path [pattern=north east lines] plot [smooth] (A)--(B)--(C);
		\end{tikzpicture}
	\caption{Newton polytope $\mathcal{N}(f)$}
	\end{minipage}
	\begin{minipage}[h]{0.49\columnwidth}
		\centering
		\begin{tikzpicture}
			\label{chapter03:figure:affineNewtonPolyhedronExample}
			% x軸y軸の描画
			\draw[->,>=stealth,semithick] (-1,0)--(4,0)node[above]{$x$};
			\draw[->,>=stealth,semithick] (0,-1)--(0,4)node[right]{$y$};
			% 点の定義
			\coordinate (B) at (1, 3);
			\coordinate (C) at (3, 2);
			\coordinate (D) at (-1, 3);
			\coordinate (E) at (3, -1);
			\coordinate (F) at (-1, -1);
			\coordinate (internal_point) at (1.5, 2);
			\coordinate (explanation_point) at (2, 3);
			% 点の描画
			\draw (0,0) node[below right]{$O$};
			\draw (B) node [above, circle] {$(1, 3)$};
			\draw (C) node [right, circle] {$(3, 2)$};
			\draw (explanation_point) node [above right] {$\mathcal{N}_{\mathrm{aff}}(f)$};
			\fill (B) circle (2pt) (C) circle (2pt);
			% 線の描画
			\draw (B)--(C);
			\draw (D)--(B);
			\draw (C)--(E);
			\draw [<-] (internal_point)--(explanation_point);
			% 斜線の描画
			\path [pattern=north east lines] plot [smooth] (D)--(B)--(C)--(E)--(F);
		\end{tikzpicture}
		\caption{affine Newton polyhedron $\mathcal{N}_{\mathrm{aff}}(f)$}
	\end{minipage}
\end{figure}
図\ref{chapter03:figure:NewtonPolytopeExample}のように,Newton polytopeは単に多項式$f$に含まれる項の指数ベクトルを頂点とした集合となっている.
対して,図\ref{chapter03:figure:NewtonPolytopeExample}のように,affine Newton polyhedronでは,$\mathcal{N}(f)$と$\mathbb{R}^2_{-}$とのMinkowski和を取ることで,"左下"全体を含んだ領域を取る無限集合となっているが,頂点に着目すると,それに伴って"左下"にあった頂点$(1, 1)$がなくなっている.これは,多項式の中で「先頭項になりそうにない次数の低い項」を取り除く操作に対応できる.この例では,項$xy$は項順序の定義から先頭項にはならず,affine Newton polyhedronを取ることでそのような項を検出することができることを意味している.

